\section{术语}\label{222}

\noindent 被在本规则中一直使用的标准术语是被意大利体化的\footnote{意大利体是一种斜体。为编纂方便,仅在本节中意大利体化标准术语。}。\\
Standard definitions used throughout these Rules are italicised.

\subsection{局}\label{2221}

\noindent 斯诺克的一\textit{局}由某比赛时段构成,此比赛时段从开始(参见第\ref{223}节规则\ref{2233}\ref{2233c})起,到每位球员[依次]在[自己的]\textit{击球轮}中[上台]比赛直到此\textit{局}结束并且是由于:\\
A \textit{frame} of snooker comprises the period of play from the start (see Section \ref{223} Rule \ref{2233}\ref{2233c}), each player playing in \textit{turn} until the \textit{frame} is completed by:
\begin{enumerate}[label=(\alph*)]
    \item 由任意[一方]球员[提出]的被[对方]接受的认输;或\\
    an accepted concession by any player; or
    \item 当黑球是台面上仅剩的\textit{目标球}, 累计得分[与比赛结果]无关, 并且[双方的]分数间有多于七分的差距且\textit{击球手}[方]领先时,由\textit{击球手}[提出]的[己方赢得此局的]申明;或\\
    a claim by the \textit{striker}, when Black is the only \textit{object ball} remaining on the table, aggregate points are not relevant, and there is a difference of more than seven points between the scores in the \textit{striker}'s favour; or
    \item 当黑球是台面上仅剩的\textit{目标球}, 累计得分[与比赛结果]无关, 并且[双方的]分数间有多于七分的差距且非\textit{击球手}[方]领先时,被判定非\textit{击球手}[方]赢得;或\\
    being awarded to the non-\textit{striker}, when Black is the only \textit{object ball} remaining on the table, aggregate points are not relevant, and there is a difference of more than seven points between the scores in the non-\textit{striker}'s favour; or
    \item 当黑球是台面上仅剩的\textit{目标球}时,第一次\textit{击球入袋}或\textit{犯规}(参见第\ref{223}节规则\ref{2234});或\\
    the first \textit{pot} or \textit{foul}, when Black is the only \textit{object ball} remaining on the table (see Section \ref{223} Rule \ref{2234}); or
    \item 被裁判根据第\ref{223}节规则\ref{22314}\ref{22314d}\ref{22314dii}或第\ref{224}节规则\ref{2241}\ref{2241b}、\ref{2241}\ref{2241d}、\ref{2243}\ref{2243b}或\ref{2243}\ref{2243c}判定[某方]赢得。\\
    being awarded by the referee under Section \ref{223} Rule \ref{22314}\ref{22314d} \ref{22314dii} or Section \ref{224} Rule \ref{2241}\ref{2241b}, \ref{2241}\ref{2241d}, \ref{2243}\ref{2243b} or \ref{2243}\ref{2243c}.
\end{enumerate}

\subsection{场}

\noindent 一\textit{场}是被商定好的或被规定的若干\textit{局}。\\
A \textit{game} is an agreed or stipulated number of \textit{frames}.

\subsection{比赛}

\noindent 一次\textit{比赛}是被商定好的或被规定的若干\textit{场}。\\
A \textit{match} is an agreed or stipulated number of \textit{games}.

\subsection{球}

\begin{enumerate}[label=(\alph*)]
    \item 白球是\textit{母球}。\\
    The White ball is the \textit{cue-ball}.
    \item 15颗红球和6颗彩球是\textit{目标球}。\\
    The 15 Reds and the 6 colours are the \textit{object balls}.
\end{enumerate}

\subsection{击球手和击球轮}

\noindent 即将[上台]比赛或已经[上台]比赛的人是\textit{击球手}。他们的\textit{击球轮}存在直到:\\
The person about to play or in play is the \textit{striker}. It is their \textit{turn} until:
\begin{enumerate}[label=(\alph*)]
    \item 一次\textit{击球}被进行并且没有分数被得到;或者\\
    a \textit{stroke} is played and no points are scored; or
    \item 一次\textit{犯规}被造成,所有球都已静止,并且裁判认可\textit{击球手}已离开球台;或者\\
    a \textit{foul} is committed, all balls have come to rest, and the referee is satisfied that the \textit{striker} has left the table; or
    \item 在一次\textit{犯规}后一个对对手的去再次击打的要求被做出;或者\\
    a request is made to the opponent to play again following a \textit{foul}; or
    \item 当黑球是台面上仅剩的\textit{目标球}, 累计得分[与比赛结果]无关, 并且[双方的]分数间有多于七分的差距且\textit{击球手}[方]领先时,本局被\textit{击球手}申明[由己方赢得];或者\\
    the frame is claimed by the \textit{striker}, when Black is the only \textit{object ball} remaining on the table, aggregate points are not relevant, and there is a difference of more than seven points between the scores in the \textit{striker}'s favour; or
    \item 最后的黑球被\textit{击球入袋}并且\textit{母球}已静止。\\
    the final Black is \textit{potted} and the \textit{cue-ball} has come to rest.
\end{enumerate}

\subsection{击球}

\begin{enumerate}[label=(\alph*)]
    \item 除瞄准并准备击打\textit{母球}(被称为运杆)时[球杆的皮头碰到\textit{母球}]外,当\textit{击球手}用球杆的皮头\textit{击打}\textit{母球}时一次\textit{击球}被进行。\\
    A \textit{stroke} is made when the \textit{striker} \textit{strikes} the \textit{cue-ball} with the tip of the cue, except while addressing the \textit{cue-ball} (known as feathering).
    \item \textit{母球}必须仅被\textit{击打}一次且禁止被向前\textit{推}。在\textit{母球}开始移动后球杆的皮头可以短暂地与其保持接触。\\
    The \textit{cue-ball} must be \textit{struck} only once and not \textit{pushed} forward. The tip of the cue may momentarily remain in contact with the \textit{cue-ball} after it commences motion.
    \item 当没有本规则中的违规被造成时一次击球是合法的。\\
    A \textit{stroke} is legal when no \textit{infringement} of these Rules is committed.
    \item 一次\textit{击球}没有结束直到:\\
    A \textit{stroke} is not completed until:
    \begin{enumerate}[label=(\roman*)]
        \item 所有球都已静止;\\
        all balls have come to rest;
        \item 有必要的任何球的摆上点位都已完成;并且\\
        spotting of any balls required is completed; and
        \item 任何正被\textit{击球手}使用的辅助器材都已被拿走,或裁判认可此次\textit{击球}已结束。\\
        any ancillary equipment being used by the \textit{striker} has been removed, or the referee is satisfied that the \textit{stroke} is completed.
    \end{enumerate}
    \item 一次\textit{击球}可以被\textit{直接地}或\textit{间接地}进行,从而:\\
    A \textit{stroke} may be made \textit{directly} or \textit{indirectly}, thus:
    \begin{enumerate}[label=(\roman*)]
        \item 当\textit{母球}没有首先击中库边就击中一颗\textit{目标球}时\textit{击球}是\textit{直接的};\\
        a \textit{stroke} is \textit{direct} when the \textit{cue-ball} hits an \textit{object ball} without first hitting a cushion;
        \item 当\textit{母球}在击中一颗\textit{目标球}前有击中一个或多个库边时\textit{击球}是\textit{间接的}。\\
        a \textit{stroke} is \textit{indirect} when the \textit{cue-ball} hits one or more cushions before hitting an \textit{object ball}.
    \end{enumerate}
\end{enumerate}

\subsection{击球入袋和掉袋}

\noindent 一次\textit{入袋}是一颗\textit{目标球}与另一颗球接触后进入袋中且无任何本规则中的\textit{违规}。造成一颗球被\textit{击球入袋}被称为\textit{击球入袋}。在一次\textit{犯规的}\textit{击球}中造成一颗球进入球袋被称为\textit{掉袋}。\\
A \textit{pot} is when an \textit{object ball}, after contact with another ball and without any \textit{infringement} of these Rules, enters a pocket. Causing a ball to be \textit{potted} is known as \textit{potting}. Causing a ball to enter a pocket in a \textit{foul} \textit{stroke} is known as \textit{pocketing}.

\subsection{单杆}

\noindent 一次\textit{单杆}是由\textit{击球手}在任意一个\textit{击球轮}中进行的连续\textit{击球}中的若干\textit{入袋}。\\
A \textit{break} is a number of \textit{pots} in successive \textit{strokes} made in any one \textit{turn} by the \textit{striker}.

\subsection{手中球状态}

\begin{enumerate}[label=(\alph*)]
    \item \textit{母球}处于\textit{手中球状态}:\\
    The \textit{cue-ball} is \textit{in-hand}:
    \begin{enumerate}[label=(\roman*)]
        \item 在每\textit{局}的开始前;\\
        before the start of each \textit{frame};
        \item 当它已\textit{掉袋}时;\\
        when it has been \textit{pocketed};
        \item 当它已被\textit{迫离台面}时;或\\
        when it has been \textit{forced off the table}; or
        \item 当黑球如第\ref{223}节规则\ref{2234}\ref{2234b}中[所述]地被重置时。\\
        when the Black is re-spotted as in Section \ref{223} Rule \ref{2234}\ref{2234b}.
    \end{enumerate}
    \item \textit{母球}保持\textit{手中球状态}直到:\\
    The \textit{cue-ball} remains \textit{in-hand} until:
    \begin{enumerate}[label=(\roman*)]
        \item 它被从\textit{手中球状态}起合法地击打;或者\\
        it is played legally from \textit{in-hand}; or
        \item 当\textit{母球}不在\textit{击球手}的控制中时一次\textit{犯规}被造成并与它有关。\\
        a \textit{foul} is committed involving the \textit{cue-ball} while it is not in the \textit{striker}'s possession.
    \end{enumerate}
    \item 当\textit{母球}处于如上面[所述]的\textit{手中球状态}时\textit{击球手}[也]被称为处于\textit{手中球状态}。\\
    The \textit{striker} is said to be \textit{in-hand} when the \textit{cue-ball} is \textit{in-hand} as above.
\end{enumerate}

\subsection{处于比赛中状态的球}

\begin{enumerate}[label=(\alph*)]
    \item 当\textit{母球}不处于\textit{手中球状态}时它处于\textit{比赛中状态}。\\
    The \textit{cue-ball} is \textit{in play} when it is not \textit{in-hand}.
    \item 自\textit{局}的开始起直到被\textit{击球入袋}、\textit{掉袋}或被\textit{迫离台面}\textit{目标球}[都]处于\textit{比赛中状态}。\\
    \textit{Object balls} are \textit{in play} from the start of the \textit{frame} until \textit{potted}, \textit{pocketed} or \textit{forced off the table}.
    \item 当被重置时彩球重新变成处于\textit{比赛中状态}。\\
    Colours become \textit{in play} again when re-spotted.
\end{enumerate}

\subsection{活球}

\noindent \textit{活球}是任何可以因\textit{母球}的首次撞击而被合法地击中的球,或任何虽不可以被如此击中但可以被[合法地]\textit{击球入袋}的球。\\
A \textit{ball on} is any ball which may be legally hit by the first impact of the \textit{cue-ball}, or any ball which may not be so hit but which may be \textit{potted}.

\subsection{指定球}

\begin{enumerate}[label=(\alph*)]
    \item \textit{指定球}是\textit{击球手}合裁判的认可地指明的,或指定(口头声明)的,他们承诺要通过\textit{母球}的首次撞击击中的\textit{目标球}。\\
    A \textit{nominated ball} is the \textit{object ball} which the \textit{striker} indicates to the satisfaction of the referee, or declares (states verbally), they undertake to hit with the first impact of the \textit{cue-ball}.
    \item 如果被裁判要求,\textit{击球手}必须指定他们以哪颗球为\textit{活}[球]。\\
    If requested by the referee, the \textit{striker} must declare which ball they are \textit{on}.
\end{enumerate}

\subsection{自由球}

\noindent \textit{自由球}是当[\textit{母球}]在一次\textit{犯规}后被\textit{做斯诺克}(参见第\ref{223}节规则\ref{22312})时\textit{击球手}\textit{指定}当成\textit{活球}的不是\textit{活球}的球。\\
A \textit{free ball} is a ball, other than the \textit{ball on}, which the \textit{striker} \textit{nominates} as the \textit{ball on} when \textit{snookered} after a \textit{foul} (see Section \ref{223} Rule \ref{22312}).

\subsection{被迫离台面}

\noindent 如果一颗球静止但不在比赛区域中或球袋中那么它被\textit{迫离台面}。\\
A ball is \textit{forced off the table} if it comes to rest other than on the 
playing area or in a pocket.

\subsection{违规、犯规和受罚}

\noindent \textit{违规}是任意对本规则的违背。\textit{犯规}是会结束违规方\textit{击球轮}的\textit{违规}。\textit{受罚}是不影响比赛次序的\textit{违规}。\\
An \textit{infringement} is any violation of these Rules. A \textit{foul} is an \textit{infringement} which will end the offender's \textit{turn}. \textit{Penalties} are \textit{infringements} which do not affect the order of play.

\subsection{罚分}

\noindent 在任何\textit{违规}后\textit{罚分}都被加给非违规方。\\
\textit{Penalty points} are awarded to the non-offender after any \textit{infringement}.

\subsection{被做斯诺克}\label{22217}

\noindent 当对每颗\textit{活球}的沿直线的\textit{直接击球}都被一颗或多颗非\textit{活}球完全或部分阻挡时\textit{母球}被\textit{做斯诺克}。如果一颗或更多颗\textit{活球}不受任何非\textit{活}球的阻挡影响而两个薄边都能被击中,那么\textit{母球}没有被\textit{做斯诺克}。\\
The \textit{cue-ball} is \textit{snookered} when a \textit{direct stroke} in a straight line to every \textit{ball on} is wholly or partially obstructed by a ball or balls not \textit{on}. If one or more \textit{balls on} can be hit at both extreme edges free of obstruction by any ball not \textit{on}, the \textit{cue-ball} is not \textit{snookered}.
\begin{enumerate}[label=(\alph*)]
    \item 如果[\textit{母球}]处于\textit{手中球状态},那么若\textit{母球}在D形区边线上或边线内的任何[被摆放的]可能位置都如上面所述地被阻挡则它被\textit{做斯诺克}。\\
    If \textit{in-hand}, the \textit{cue-ball} is \textit{snookered} if it is obstructed as described above from all possible positions on or within the lines of the ``D''.
    \item 如果\textit{母球}被多于一颗非\textit{活}球如此阻挡而不能击中\textit{活球}那么:\\
    If the \textit{cue-ball} is so obstructed from hitting a \textit{ball on} by more than one ball not \textit{on}:
    \begin{enumerate}[label=(\roman*)]
        \item 最接近\textit{母球}的球被认为是有效障碍球;并且\\
        the ball nearest to the \textit{cue-ball} is considered to be the effective snookering ball; and
        \item 万一多于一颗阻挡的球和\textit{母球}距离相同,所有这些球都会被认为是有效障碍球。\\
        should more than one obstructing ball be equidistant from the \textit{cue-ball}, all such balls will be considered to be effective snookering balls.
    \end{enumerate}
    \item 当红球是\textit{活球}时,如果\textit{母球}被不同非\textit{活}球阻挡而不能击中不同红球,那么没有有效障碍球。\\
    When Red is the \textit{ball on}, if the \textit{cue-ball} is obstructed from hitting different Reds by different balls not \textit{on}, there is no effective snookering ball.
    \item 当\textit{母球}如上面[所述]地被\textit{做斯诺克}时\textit{击球手}[也]被称为被\textit{做斯诺克}。\\
    The \textit{striker} is said to be \textit{snookered} when the \textit{cue-ball} is \textit{snookered} as above.
    \item \textit{母球}不能被库边\textit{做斯诺克}。\\
    The \textit{cue-ball} cannot be \textit{snookered} by a cushion.
\end{enumerate}

\subsection{被占的点位}

\noindent 如果某一颗球不能被摆放到一个点位上并避免此球触碰另一颗球那么它被称为被\textit{占}。\\
A spot is said to be \textit{occupied} if a ball cannot be placed on it without that ball touching another ball.

\subsection{推击}

\noindent 一次\textit{推击}被造成于某时,此时球杆的皮头与\textit{母球}保持接触且保持接触时:\\
A \textit{push stroke} is made when the tip of the cue remains in contact with the \textit{cue-ball};
\begin{enumerate}[label=(\alph*)]
    \item \textit{母球}已经开始其移动,除非是[球杆的皮头与\textit{母球}]在初次接触的时段短暂地[保持接触];或\\
    after the \textit{cue-ball} has commenced its motion, other than momentarily at the point of initial contact; or
    \item \textit{母球}接触一颗\textit{目标球},但有例外是当\textit{母球}和一颗\textit{目标球}几乎相贴时,如果\textit{母球}击中此\textit{目标球}的极薄边,那么这不应被认为是一次\textit{推击}。\\
    as the \textit{cue-ball} contacts an \textit{object ball} except, where the \textit{cue-ball} and an \textit{object ball} are almost touching, it shall not be deemed a \textit{push stroke} if the \textit{cue-ball} hits a very fine edge of the \textit{object ball}.
\end{enumerate}

\subsection{跳球}

\noindent 一次\textit{跳球}被造成于某时,此时\textit{母球}越过一颗\textit{目标球}的任意部分,且无论在此过程中是否将其击中,除非是:\\
A \textit{jump shot} is made when the \textit{cue-ball} passes over any part of an \textit{object ball}, whether hitting it in the process or not, except:
\begin{enumerate}[label=(\alph*)]
    \item \textit{母球}首先击中一颗不是与之相贴的球的\textit{目标球},然后跳起越过另一颗球;或\\
    when the \textit{cue-ball} first hits one \textit{object ball}, other than a touching ball, and then jumps over another ball; or
    \item \textit{母球}跳起并击中一颗不是与之相贴的球的\textit{目标球},并且在落到比赛区域的瞬间,\textit{母球}没有落在此\textit{目标球}当前位置的远端;或\\
    when the \textit{cue-ball} jumps and hits an \textit{object ball}, other than a touching ball, and at the moment of landing on the playing area, the \textit{cue-ball} is not on the far side of the current position of that \textit{object ball}; or
    \item 合法地击中一颗不是与之相贴的球的\textit{目标球}后,\textit{母球}击中库边或另一颗球后再跳起越过此[\textit{目标}]\textit{球}。\\
    when, after legally hitting an \textit{object ball}, other than a touching ball, the \textit{cue-ball} jumps over that ball after hitting a cushion or another ball.
\end{enumerate}

\subsection{未尽力}

\noindent 一次\textit{未尽力}是:\\
A \textit{miss} is:
\begin{enumerate}[label=(\alph*)]
    \item \textit{母球}未能首先接触一颗\textit{活球};或\\
    when the \textit{cue-ball} fails to first contact a \textit{ball on}; or
    \item 当\textit{自由球}已被\textit{指定}时,\textit{母球}既未能首先击中\textit{被指定的}\textit{自由球}也未能同时击中此球和一颗\textit{活球}。\\
    when a \textit{free ball} has been \textit{nominated}, the \textit{cue-ball} fails to first hit the \textit{nominated} \textit{free ball} or that ball simultaneously with a \textit{ball on}.
\end{enumerate}

\subsection{磋商时间}

\noindent \textit{磋商时间}是球员可以就将任意球摆回到\textit{违规}被造成(第\ref{223}节规则\ref{2232}\ref{2232c}\ref{2232cii}、\ref{2233}\ref{2233k}、\ref{22310}\ref{22310i}、\ref{22314}、\ref{22315}和\ref{22316})前的或如第\ref{223}节规则\ref{2239}中所述的原来的位置,对裁判给予帮助的时间。\textit{磋商时间}在将球摆回的决定被做出时开始,并在双方球员都对球的位置表示认可时结束或依裁判的最终决定结束。\\
A \textit{consultation period} is the period in which players may assist the referee with replacing any ball(s) to the original position prior to when an \textit{infringement} was committed (Section \ref{223} Rules \ref{2232}\ref{2232c}\ref{2232cii}, \ref{2233}\ref{2233k}, \ref{22310}\ref{22310i}, \ref{22314}, \ref{22315} and \ref{22316}) or as described in Section \ref{223} Rule \ref{2239}. The \textit{consultation period} starts from the moment the decision is made to replace the ball(s) and ends when both players are satisfied as to the position of the ball(s), or by the referee's final decision.

\section{比赛}\label{223}

\subsection{总述}\label{2231}

\noindent 斯诺克可以由两位球员独立参赛,或多于两位球员分为两方参赛。比赛在下面的段落\ref{2231a}至\ref{2231h}中概述。\\
Snooker may be played by two players independently, or by more than two players as sides. The Game is summarised in paragraphs \ref{2231a} to \ref{2231h} below.
\begin{enumerate}[label=(\alph*)]
    \item \label{2231a}每位球员都使用相同的白色母球且一共有二十一颗目标球——十五颗每颗1分的红球,以及六颗彩球:2分的黄球、3分的绿球、4分的棕球、5分的蓝球、6分的粉球和7分的黑球。\\
    Each player uses the same White cue-ball and there are twenty-one object balls --- fifteen Reds each valued 1, and six colours: Yellow valued 2, Green 3, Brown 4, Blue 5, Pink 6 and Black 7.
    \item 球员的击球轮内的得分击球被通过交替地将红球和彩球击球入袋直到所有红球都离开台面和之后[将]彩球按其分值从低到高的顺序[击球入袋]进行。\\
    Scoring strokes in a player's turn are made by potting Reds and colours alternately until all the Reds are off the table and then the colours in the ascending order of their value.
    \item 因得分击球而得到的分数被加到击球手[方]的得分上。\\
    Points awarded for scoring strokes are added to the score of the striker.
    \item 因违规导致的罚分被加到对方的得分上。\\
    Penalty points from infringements are added to the opponent's score.
    \item 一个在整局内的任何时候都被采用的战术是让母球留在一颗非活球的后面以致其对下一位球员而言被做斯诺克。如果一位球员或一方需要[获得]比通过台面上剩余的球能得到的[分数]更多的分数[才能获胜],那么做斯诺克期待因[对方的]犯规而获得分数变得最为重要。\\
    A tactic employed at any time during a frame is to leave the cue-ball behind a ball not on such that it is snookered for the next player. If a player or side requires more points than are available from the balls remaining on the table, then the laying of snookers in the hope of gaining points from fouls becomes most important.
    \item 一局的胜者是某球员或某方:\\
    The winner of a frame is the player or side:
    \begin{enumerate}[label=(\roman*)]
        \item 有最高分数的;\\
        with the highest score;
        \item 此局[对方]认输的;或\\
        to whom the frame is conceded; or
        \item 此局被[裁判]根据第\ref{223}节的规则\ref{22314}\ref{22314d}\ref{22314dii}或第\ref{224}节的规则\ref{2241}\ref{2241b}、\ref{2241}\ref{2241d}、\ref{2243}\ref{2243b}或\ref{2243}\ref{2243c}判定[己方]赢得的。\\
        to whom the frame is awarded under Section \ref{223} Rule \ref{22314}\ref{22314d}\ref{22314dii} or Section \ref{224} Rule \ref{2241}\ref{2241b}, \ref{2241}\ref{2241d}, \ref{2243}\ref{2243b} or \ref{2243}\ref{2243c}.
    \end{enumerate}
    \item 一场的胜者是某球员或某方:\\
    The winner of a game is the player or side:
    \begin{enumerate}[label=(\roman*)]
        \item 赢得最多的或要求的若干局的;\\
        winning the most, or required, number of frames;
        \item 当累计得分是[胜负]依据时获得最高总分的;或\\
        making the greatest total where aggregate points are relevant; or
        \item 此场被[裁判]根据第\ref{224}节规则\ref{2241}\ref{2241c}或\ref{2241}\ref{2241d}判定[己方]赢得的。\\
        to whom the game is awarded under Section \ref{224} Rule \ref{2241}\ref{2241c} or \ref{2241}\ref{2241d}.
    \end{enumerate}
    \item \label{2231h}一次比赛的胜者是某球员或某方赢得最多场的或当累计得分是[胜负]依据时有最高总分的。\\
    The winner of a match is the player or side winning the most games or, where aggregate points are relevant, with the greatest total.
\end{enumerate}

\subsection{球的位置}\label{2232}

\begin{enumerate}[label=(\alph*)]
    \item 在每局的开始,母球处于手中球状态并且目标球如下面[所述]地放置在台面上:\\
    At the start of each frame, the cue-ball is in-hand and the object balls are positioned on the table as follows:
    \begin{enumerate}[label=(\roman*)]
        \item ${}$[全部]红球呈一个紧密相贴的等边三角形,且顶端的红球位于球桌纵向中线上,粉球点上方以致其会尽可能接近粉球点但不占之,并且[红球]三角形的底边与顶库平行;\\
        the Reds in the form of a tightly-packed equilateral triangle, with the Red at the apex situated on the centre longitudinal line of the table, above the Pink Spot such that it will be as close to the Pink Spot as possible without occupying it, and the base of the triangle parallel with the Top Cushion;
        \item 六颗彩球在第\ref{221}节规则\ref{2211}\ref{2211f}中所设定的[各自的]点位上。\\
        the six colours on the spots designated in Section \ref{221} Rule \ref{2211}\ref{2211f}.
    \end{enumerate}
    \item 如果一个台面初设中的错误被造成,那么第\ref{223}节规则\ref{2237}\ref{2237c} 应适用,此局如第\ref{223}节规则\ref{2233}\ref{2233c}中[所述]地开始。\\
    If an error in setting up the table is made, Section \ref{223} Rule \ref{2237}\ref{2237c} shall apply, the frame starting as in Section \ref{223} Rule \ref{2233}\ref{2233c}.
    \item \label{2232c}在一局已开始后,处于比赛中状态的球只可以被裁判依击球手的合理请求\footnote{如果可能由于将某球拿起清洁并摆回后位置的微小误差,导致本不能被合法地直接击中或击球入袋的球变得能被合法地直接击中或击球入袋,那么原则上此时清洁此球的请求不是合理请求。}清洁并且:\\
    After a frame has started, a ball in play may only be cleaned by the referee upon reasonable request by the striker and:
    \begin{enumerate}[label=(\roman*)]
        \item 在为清洁而球被拿走前此球的位置应被用合适的装置标注;\\
        the position of the ball shall be marked by a suitable device prior to the ball being lifted for cleaning;
        \item \label{2232cii}此被用于标注正被清洁的球的位置的装置应被视为此球并得到此球的分值直至此球已被清洁完成且已被摆回完成之时。如果任意球员接触此装置那么是违规。裁判应将相应的罚分加给非违规方并且若有必要则将此装置或此正被清洁球摆回它原来的位置且即使它被拿起[也如此处理]。\\
        the device used to mark the position of a ball being cleaned shall be regarded as, and acquire the value of, the ball until such time as the ball has been cleaned and replaced. It is an infringement if any player contacts the device. The referee shall award the relevant penalty points to the non-offender and replace the device or ball being cleaned to its original position, if necessary, even if it was picked up.

        当将装置或球摆回的决定被做出时磋商时间开始。\\
        A consultation period starts when the decision is made to replace the device or ball.
    \end{enumerate}
\end{enumerate}

\subsection{比赛方式}\label{2233}

\noindent 球员们应以抽签或任意双方都同意的方式决定比赛次序,[抽签的]胜者有哪位球员首先击打的选择权。\\
The players shall determine the order of play by lot or in any mutually agreed manner, the winner having the choice of which player plays first.
\begin{enumerate}[label=(\alph*)]
    \item 被如此决定的比赛次序必须在整局内保持不变,但有例外是球员在任何[自己造成的]犯规后都可以被下一位球员要求再次击打。\\
    The order of play thus determined must remain unaltered throughout the frame, except that a player may be asked by the next player to play again after any foul.
    \item 在整场内每局第一个击打的某球员或某方必须交替。\\
    The player or side to play first must alternate for each frame during a game.
    \item \label{2233c}第一个[击打的]球员从手中球状态起击打,此局开始于某时,此时母球已被摆放到比赛区域中并被球杆的皮头接触且被接触时:\\
    The first player plays from in-hand, the frame commencing when the cue-ball has been placed on the playing area and contacted by the tip of the cue either:
    \begin{enumerate}[label=(\roman*)]
        \item 一次击球被进行;或\\
        as a stroke is made; or
        \item ${}$[球员]正在瞄准母球。\\
        while addressing the cue-ball.
    \end{enumerate}
    \item \label{2233d}如果一局被错误的某球员或某方开启那么:\\
    If a frame is started by the wrong player or side:
    \begin{enumerate}[label=(\roman*)]
        \item 若仅仅一次击球已被进行且自此没有违规已被造成,则其应被正确地重新开启,且无处罚;或者\\
        it shall be re-started correctly, without penalty, if only one stroke has been played and no infringement has been committed since; or
        \item 若另一次击球已被进行或一次违规在第一次击球中或第一次击球完成后被造成,则其应按正常方式继续,且开[局]的正确次序应在下一局中被恢复以致某球员或某方将在连续三局中开[局]完成;或者\\
        it shall continue in the normal way if another stroke is made, or if an infringement is committed during the first stroke or after the completion of the first stroke, with the correct order of starting being resumed in the following frame such that one player or side will have started in three consecutive frames; or
        \item \label{2233diii}若僵局事件被宣布(参见第\ref{223}节规则\ref{22317}),则其应被正确的某球员或某方重新开启。\\
        it shall, in the event of a stalemate being declared (see Section \ref{223} Rule \ref{22317}), be re-started by the correct player or side.
    \end{enumerate}
    \item 对一次合法的击球,所有第\ref{223}节规则\ref{22311}中所述的违规都禁止出现。\\
    For a stroke to be legal, none of the infringements described in Section \ref{223} Rule \ref{22311} must occur.
    \item 确保所有当前击球轮或之前击球轮中[用到]的物品或辅助器材被移除于台面是击球手的责任。\\
    It is the striker's responsibility to ensure that all objects or ancillary equipment from this turn or previous turns are removed from the table.
    \item 对每个击球轮的第一次击球,红球或某被指定当成红球的自由球是活球,并且在相同的击球内被击球入袋的每个红球和任何被指定当成红球的自由球的分数都被记录,直到所有红球都离开台面。\\
    For the first stroke of each turn, until all Reds are off the table, Red or a free ball nominated as a Red is the ball on, and the value of each Red and any free ball nominated as a Red, potted in the same stroke, is scored.
    \item \label{2233h}
    \begin{enumerate}[label=(\roman*)]
        \item 如果一颗红球或一颗被指定当成红球的自由球被击球入袋,那么相同的球员进行下一次击球并且下一颗活球是一颗依击球手选择的彩球,若该彩球被击球入袋则其被计分且彩球随后被摆上点位。\\
        If a Red, or a free ball nominated as a Red, is potted, the same player plays the next stroke and the next ball on is a colour of the striker's choice which, if potted, is scored and the colour is then spotted.
        \item 单杆被通过交替地将红球和彩球击球入袋延续直到所有红球都离开台面并且届时一颗彩球已在最后一颗红球的击球入袋后被击打。\\
        The break is continued by potting Reds and colours alternately until all the Reds are off the table and, where applicable, a colour has been played at following the potting of the last Red.
        \item \label{2233hiii}彩球随后根据第\ref{223}节规则\ref{2231}\ref{2231a}按其分值从低到高的顺序成为活[球]且在下一次被击球入袋时保持离开台面,除非是如第\ref{223}节规则\ref{2234}中所规定的[情形],并且击球手对下一颗活[球]彩球进行下一次击球。\\
        The colours then become on in the ascending order of their value as per Section \ref{223} Rule \ref{2231}\ref{2231a} and when next potted remain off the table, except as provided for in Section \ref{223} Rule \ref{2234}, and the striker plays the next stroke at the next colour on.
        \item 如果在某次单杆中的击球手在裁判已在其他所有球都静止时完成将某彩球摆上点位前击打,那么该彩球的分数不应被记录并且第\ref{223}节规则\ref{22311}\ref{22311a}\ref{22311ai}或\ref{22311} \ref{22311b}\ref{22311bii}应视情况适用。\\
        In the event that the striker, in a break, plays before the referee has completed spotting a colour while all other balls are at rest, the value of the colour shall not be scored and Section \ref{223} Rule \ref{22311}\ref{22311a} \ref{22311ai} or \ref{22311}\ref{22311b}\ref{22311bii} shall apply as appropriate.
    \end{enumerate}
    \item 一旦被击球入袋、掉袋或被迫离台面,通常红球就不会被摆回台面且不考虑某球员可能会因此从一次犯规中获益的事实。然而,例外在第\ref{223}节规则\ref{2232}\ref{2232c}\ref{2232cii}、\ref{2233}\ref{2233k}、\ref{2239}、\ref{22310}\ref{22310i}、\ref{22314}\ref{22314b}、\ref{22314}\ref{22314e}、\ref{22315}、\ref{22316}、\ref{22320}\ref{22320b}和第\ref{225}节规则\ref{2251}\ref{2251a}\ref{2251ai}中被规定。\\
    Reds are not usually replaced on the table once potted, pocketed or forced off the table regardless of the fact that a player may thus benefit from a foul. However, exceptions are provided for in Section \ref{223} Rules \ref{2232}\ref{2232c}\ref{2232cii}, \ref{2233}\ref{2233k}, \ref{2239}, \ref{22310}\ref{22310i}, \ref{22314}\ref{22314b}, \ref{22314}\ref{22314e}, \ref{22315}, \ref{22316}, \ref{22320}\ref{22320b} and Section \ref{225} Rule \ref{2251}\ref{2251a}\ref{2251ai}.
    \item 如果击球手未能将球击球入袋,那么他们必须下台且不带不适当的耽搁。如果他们在下台前或下台时造成任意犯规,那么他们会被如第\ref{223}节规则\ref{22311}中所规定地处罚。下一次击球被在母球静止处随后进行,或若母球不处于比赛中状态则被从手中球状态起进行,除非当母球被根据第\ref{223}节规则\ref{22310}\ref{22310i}、\ref{22314}\ref{22314e}和\ref{22316}摆回时。\\
    If the striker fails to pot a ball, they must leave the table without undue delay. In the event that they should commit any foul before, or while leaving the table, they will be penalised as provided for in Section \ref{223} Rule \ref{22311}. The next stroke is then played from where the cue-ball comes to rest, or from in-hand if the cue-ball is not in play, except when the cue-ball is replaced in accordance with Section \ref{223} Rules \ref{22310}\ref{22310i}, \ref{22314}\ref{22314e} and \ref{22316}.
    \item \label{2233k}如果非击球手在[自己的]击球轮外上台并造成任意违规那么是受罚。裁判应宣告``受罚''并且任何被移动的球都应被摆回其违规前的位置,并且击球手的击球轮会不受影响地继续。\\
    It is a penalty if the non-striker comes to the table, out of turn, and commits any infringement. The referee shall call PENALTY and any ball(s) moved shall be replaced to their position prior to the infringement, and the striker's turn will continue unaffected.

    当将球摆回的决定被做出时磋商时间开始。\\
    A consultation period starts when the decision is made to replace the ball(s).
    \item 在对方击球轮的最后一次击球后,或一次犯规后,如果正上[台]的球员在[所有]球[都]已静止前或裁判已完成将某彩球摆上点位前击打母球或造成一次违规,那么他们应被好像他们是击球手一样地处罚并且他们的击球轮会结束。\\
    Following the final stroke of the opponent's turn, or following a foul, if an incoming player strikes the cue-ball or commits an infringement before the balls have come to rest, or before the referee has completed the spotting of a colour, they shall be penalised as if they were the striker and their turn will end.
    \item 如果任意球进入球袋又反弹上比赛区域,那么它不被算作已被击球入袋或掉袋。如果此[情形]出现那么所有球员都没有补偿。\\
    If any ball enters a pocket and rebounds onto the playing area, it does not count as having been potted or pocketed. No player has redress if this occurs.
\end{enumerate}

\subsection{局、场或比赛的结束}\label{2234}

\begin{enumerate}[label=(\alph*)]
    \item \label{2234a}当黑球是台面上仅剩的目标球时,第一次入袋或犯规结束此局且例外只有下面的条件都满足:\\
    When Black is the only object ball remaining on the table, the first pot or foul ends the frame excepting only if the following conditions both apply:
    \begin{enumerate}[label=(\roman*)]
        \item 得分届时相等;和\\
        the scores are then equal; and
        \item 累计得分不是[胜负]依据。\\
        aggregate scores are not relevant.
    \end{enumerate}
    \item \label{2234b}当上面\ref{2234a}中的条件都满足时:\\
    When both conditions in \ref{2234a} above apply:
    \begin{enumerate}[label=(\roman*)]
        \item 黑球被摆上点位;\\
        the Black is spotted;
        \item 球员们为[哪位球员]下一个击打的选择权抽签;\\
        the players draw lots for choice of playing next;
        \item 下一位球员从手中球状态开始击打;并且\\
        the next player plays from in-hand; and
        \item 第一次入袋或违规结束此局。\\
        the first pot or infringement ends the frame.
    \end{enumerate}
    \item 当累计得分决定一场或一次比赛的胜者,且最后一局结束[双方]累计得分相等时,此局中的球员们应遵照上面\ref{2234b}中所陈述的习惯上被称为重置黑球的程序[决出胜者]。\\
    When aggregate scores determine the winner of a game or match, and the aggregate scores are equal at the end of the last frame, the players in that frame shall follow the procedure, commonly known as a re-spotted Black, set out in \ref{2234b} above.
\end{enumerate}

\subsection{从手中球状态开始击打}

\noindent 为从手中球状态开始击打,母球必须在D形区边线上或边线内的某个位置被球杆的皮头接触,但它可以被朝任意方向击打。\\
To play from in-hand, the cue-ball must be contacted by the tip of the cue from a position on or within the lines of the ``D'', but it may be played in any direction.
\begin{enumerate}[label=(\alph*)]
    \item 如果被问及裁判就会声明母球是否被恰当地放置(也就是说,不在D形区边线外)。\\
    The referee will state, if asked, whether the cue-ball is properly placed (that is, not outside the lines of the ``D'').
    \item 如果处于手中球状态的母球在D形区外被球杆的皮头接触,那么它被认为是被不恰当地从手中球状态开始击打。\\
    If the cue-ball, while in-hand, is contacted by the tip of the cue while outside the ``D'', it is considered as improperly played from in-hand.
    \item 如果放置母球时球杆的皮头触碰之,而裁判[根据实际情况]确信击球手不是正在尝试进行一次击球\footnote{例如击球手是正在用球杆拨动母球以移动母球至自己满意的位置。},那么[此时]母球不处于比赛中状态。\\
    If the tip of the cue should touch the cue-ball while positioning it, and the referee is satisfied that the striker was not attempting to play a stroke, then the cue-ball is not in play.
\end{enumerate}

\subsection{同时击中两颗球}

\noindent 不是两颗红球或一颗自由球和一颗活球的两颗球禁止在母球的首次撞击中被同时击中。\\
Two balls, other than two Reds or a free ball and a ball on, must not be hit simultaneously by the first impact of the cue-ball.

\subsection{将彩球摆上点位}\label{2237}

\noindent 任何被击球入袋、掉袋或被迫离台面的彩球都应被在下一次击球被进行前摆上点位,直到根据第\ref{223}节规则\ref{2233}\ref{2233h}\ref{2233hiii}最后一次被击球入袋。\\
Any colour potted, pocketed or forced off the table shall be spotted before the next stroke is made, until finally potted under Section \ref{223} Rule \ref{2233}\ref{2233h}\ref{2233hiii}.
\begin{enumerate}[label=(\alph*)]
    \item 球员不应为任何裁判未正确地将任意球摆上点位的错误担责。\\
    A player shall not be held responsible for any mistake by the referee in failing to spot any ball correctly.
    \item 如果一颗彩球在根据第\ref{223}节规则\ref{2233}\ref{2233h}\ref{2233hiii}按[分值]从低到高的顺序被击球入袋后被错误地摆上点位,那么该错误被发现时它应被[立即]移除于台面且无处罚,并且比赛应从因此产生的状态继续。如果该彩球已被击球入袋后该错误[才]被发现, 那么[若]下一次击球被进行或一次违规被在下一次击球被进行前造成后[该错误才被发现] [则]已得到的分数应算入。\\
    If a colour is spotted in error after being potted in ascending order as per Section \ref{223} Rule \ref{2233}\ref{2233h}\ref{2233hiii}, it shall be removed from the table without penalty when the error is discovered, and play shall continue from the resulting position. If the error is discovered after the colour has been potted, the points scored shall count after the next stroke is played, or after an infringement is committed prior to playing the next stroke.
    \item \label{2237c}如果在一颗或多颗球没有被正确地摆上点位时一次击球被进行,那么它们对随后的击球而言应被视为已被正确地摆上点位。任何不正确地不在台面的彩球都应被放上点位且:\\
    If a stroke is made with a ball or balls not correctly spotted, they shall be considered correctly spotted for subsequent strokes. Any colour incorrectly missing from the table shall be spotted:
    \begin{enumerate}[label=(\roman*)]
        \item 如果[彩球不正确地]不在[台面]是因之前的疏忽所致那么[不在台面]被发现时无处罚,只要此局还未根据第\ref{222}节规则\ref{2221}的措辞结束,并且比赛应从因此产生的状态继续;或\\
        without penalty when discovered if missing due to previous oversight, provided the frame has not already ended under the terms of Section \ref{222} Rule \ref{2221} and play shall continue from the resulting position; or
        \item 如果击球手在裁判能完成摆上点位前击打那么处罚。\\
        subject to penalty if the striker played before the referee was able to complete the spotting.
    \end{enumerate}
    \item 如果一颗红球被错误地摆上点位而不是一颗彩球,那么一旦发现:\\
    If a Red is spotted in error, instead of a colour, once discovered:
    \begin{enumerate}[label=(\roman*)]
        \item 若此红球能被认出[但未被击球入袋、掉袋和被迫离台面且彩球未如上面\ref{2237c}中所述地被摆上点位],则其会被移除于台面;或者\\
        if the Red can be identified it will be removed from the table; or
        \item 若此红球能被认出但已被击球入袋、掉袋或被迫离台面,或[此红球能被认出但]彩球已如上面\ref{2237c}中所述地被摆上点位,或此红球不能被认出,则此局继续并因此等效地产生一个十六红球局。如果彩球不在台面那么它应被摆上点位并且在所有情形中比赛都应从因此产生的状态继续且无处罚。\\
        if the Red can be identified but has been potted, pocketed or forced off the table, or the colour was already spotted as described in \ref{2237c} above, or if the Red cannot be identified, the frame continues thus effectively creating a sixteen Red frame. In cases where the colour is missing from the table it shall be spotted and in all cases play shall continue from the resulting position without penalty.
    \end{enumerate}
    \item 如果一颗彩球必须被摆上点位且它自身的点位被占,那么它应被摆放到分值最高的可用的点位上。\\
    If a colour has to be spotted and its own spot is occupied, it shall be placed on the highest value spot available.
    \item 如果有多于一颗要被摆上点位的彩球且它们自身的点位[都]被占,那么最高分值的球应在摆上点位的顺序中占先。\\
    If there is more than one colour to be spotted and their own spots are occupied, the highest value ball shall take precedence in order of spotting.
    \item 如果所有点位都被占,那么彩球应被摆放得尽可能接近它自身的点位且在此点位和顶库[与此点位]最接近的部分之间。\\
    If all spots are occupied, the colour shall be placed as near to its own spot as possible, between that spot and the nearest part of the Top Cushion.
    \item 在粉球和黑球[摆上点位]的情形中,如果所有点位都被占且在相应的点位和顶库[与此点位]最接近的部分之间没有可用的空间,那么彩球应被摆放得尽可能接近它自身的点位且在球桌纵向中线上。\\
    In the case of Pink and Black, if all spots are occupied and there is no available space between the relevant spot and the nearest part of the Top Cushion, the colour shall be placed as near to its own spot as possible on the centre longitudinal line of the table.
    \item 在[上述]所有情形中,彩球在被摆上点位时禁止触碰到另一颗球。\\
    In all cases, the colour when spotted must not be touching another ball.
    \item 为被恰当地摆上点位,彩球必须被摆放到本规则所述的点位上。\\
    A colour, to be properly spotted, must be placed on the spot designated in these Rules.
\end{enumerate}

\subsection{贴球}

\begin{enumerate}[label=(\alph*)]
    \item 如果一次击球结束时母球贴住一颗或多颗活球或可以成为活球的球,那么裁判应宣告``贴球''并且若有任意[球员的] [何球被贴的]疑问则应指明母球贴住哪颗或哪几颗活球。如果在一颗红球(或一颗被指定当成红球的自由球)已被击球入袋后母球贴住一颗或多颗彩球,那么裁判还应要求击球手指定他们以哪颗球为活[球]。\\
    If at the completion of a stroke the cue-ball is touching a ball or balls on, or that could be on, the referee shall call TOUCHING BALL and, in the event of any doubt, indicate which ball or balls on the cue-ball is touching. If the cue-ball is touching one or more colours after a Red (or a free ball nominated as a Red) has been potted, the referee shall also ask the striker to DECLARE which colour they are on.
    \item 当一次``贴球''已被宣告时,击球手必须将母球打离该球并不使其移动否则是一次推击。\\
    When a TOUCHING BALL has been called, the striker must play the cue-ball away from that ball without moving it or it is a push stroke.
    \item 只要在进行一次击球时击球手没有造成任何贴住的目标球移动,就不应有处罚,不过前提是:\\
    Providing the striker, in playing a stroke, does not cause any touching object ball to move, there shall be no penalty if:
    \begin{enumerate}[label=(\roman*)]
        \item ${}$[被贴住的]球是活[球]\\
        the ball is on;
        \item ${}$[被贴住的]球可以成为活[球]并且击球手指定他们以之为活[球];或者\\
        the ball could be on and the striker declares they are on it; or
        \item ${}$[被贴住的]球可以成为活[球]并且击球手指定并首先击中另一颗可以成为活[球]的球。\\
        the ball could be on and the striker declares, and first hits, another ball that could be on.
    \end{enumerate}
    \item 如果母球静止并贴住或几乎贴住一颗非活球,那么裁判如果被问及它是否贴住就会阐明情况。\\
    If the cue-ball comes to rest touching or nearly touching a ball that is not on, the referee, if asked whether it is touching, will clarify the situation.
    \item 当母球贴住一颗活球和一颗非活球时,裁判应只指明贴住的活球。如果击球手问裁判母球是否也贴住非活球,那么他们有权被告知。\\
    When the cue-ball is touching both a ball on and a ball not on, the referee shall only indicate the ball on as touching. If the striker should ask the referee whether the cue-ball is also touching the ball not on, they are entitled to be told.
    \item 如果裁判[根据实际情况]确信击球时贴住的球的任意移动不是由击球手造成的\footnote{例如击球手的动作本不可能造成贴住的球移动,但因静电,贴住的球在母球被打离后有微小移动。},那么不是犯规。\\
    It is not a foul if the referee is satisfied that any movement of a touching ball at the moment of striking was not caused by the striker.
    \item 如果一颗当被裁判检查时没有贴住母球的静止的目标球之后在一次击球已被进行前被发现和母球接触了,此球应被裁判按他们所确信的[原位]置回原位。这亦适用于一颗贴住的之后当被裁判检查时没有贴住的球。\\
    If a stationary object ball, not touching the cue-ball when examined by the referee, is later seen to be in contact with the cue-ball before a stroke has been made, the balls shall be repositioned by the referee to their satisfaction. This also applies to a touching ball which later, when examined by the referee, is not touching.
\end{enumerate}

\subsection{在袋口的球}\label{2239}

\noindent 当一颗球掉入球袋,它没有被另一颗球击中,且\\
When a ball falls into a pocket without being hit by another ball, and
\begin{enumerate}[label=(\alph*)]
    \item 这不是任意进行中的击球的一部分时,它应被摆回并且任何得到的分数都应算入。\\
    being no part of any stroke in progress, it shall be replaced and any points scored shall count.
    \item 如果它本会被一次击球中牵扯到的任意球击中:\\
    If it would have been hit by any ball involved in a stroke:
    \begin{enumerate}[label=(\roman*)]
        \item 且没有本规则中的违规(包括一次违规本会发生但因掉入球袋的球[而未发生]的情形),那么所有球都会被摆回并且相同的击球会被再次进行,或一次不同的击球可以被相同的击球手按其自行决定权进行;\\
        with no infringement of these Rules (including cases where an infringement would have occurred but for the ball falling into a pocket), all balls will be replaced and the same stroke played again, or a different stroke may be played by the same striker at their discretion;
        \item 如果一次犯规被造成,那么击球手招致第\ref{223}节规则 \ref{22311}中规定的处罚,所有球都会被摆回并且下一位球员有一次犯规后通常的选择权。\\
        if a foul is committed, the striker incurs the penalty prescribed in Section \ref{223} Rule \ref{22311}, all balls will be replaced and the next player has the usual options after a foul.
    \end{enumerate}
    \item 如果一颗球在袋口短暂地保持平衡且随后掉入[球袋],那么它应被视为被击球入袋或掉袋且不应被摆回。\\
    If a ball balances momentarily on the edge of a pocket and then falls in, it shall be considered potted or pocketed and shall not be replaced.
\end{enumerate}
\noindent 当将球摆回的决定被做出时磋商时间开始。\\
A consultation period starts when the decision is made to replace the ball(s).

\subsection{犯规}\label{22310}

\noindent 如果一次犯规被造成,那么裁判应立即宣告``犯规''\\
If a foul is committed, the referee shall immediately call FOUL.
\begin{enumerate}[label=(\alph*)]
    \item 如果击球手还未进行一次击球,那么他们的击球轮结束并且裁判应宣布处罚。\\
    If the striker has not made a stroke, their turn ends and the referee shall announce the penalty.
    \item 如果一次击球已被进行,那么裁判会在宣布处罚前等到击球完成。\\
    If a stroke has been made, the referee will wait until completion of the stroke before announcing the penalty.
    \item 如果一次犯规或一次受罚既没有被裁判判定,也没有在下一次击球被进行前被非击球手成功地申明,那么它被忽略。\\
    If a foul or a penalty is neither awarded by the referee, nor successfully claimed by the non-striker before the next stroke is made, it is condoned.
    \item 任何没有被正确地摆上点位的彩球都应保持于原位除若其离开台面则其应被正确地摆上点位。\\
    Any colour not correctly spotted shall remain where positioned except that if off the table it shall be correctly spotted.
    \item 在一次犯规被判定前的一次单杆中得到的所有分数都应算入但击球手不应在一次犯规的击球中因任何掉袋的球得到任何分数。\\
    All points scored in a break before a foul is awarded shall count but the striker shall not score any points for any ball pocketed in a foul stroke.
    \item 下一次击球被从母球静止处进行或若母球不处于比赛中状态则被从手中球状态起进行。\\
    The next stroke is played from where the cue-ball comes to rest or, if the cue-ball is not in play, from in-hand.
    \item 如果在同一击球中多于一次犯规被造成,那么分数最高的处罚应被招致。\\
    If more than one foul is committed in the same stroke, the highest value penalty shall be incurred.
    \item 造成犯规的球员:\\
    The player who committed the foul:
    \begin{enumerate}[label=(\roman*)]
        \item 招致第\ref{223}节规则\ref{22311}中规定的罚分;并且\\
        incurs the penalty points prescribed in Section \ref{223} Rule \ref{22311}; and
        \item 若被下一位球员要求则必须进行下一次击球。\\
        has to play the next stroke if requested by the next player.
    \end{enumerate}
    \item \label{22310i}如果击球手在击打母球前对包括它在内的任意球犯规,那么适当的处罚会被推行。非违规者可以随后选择自己从留存的状态击打,或要求违规者从留存的状态或原来的状态再次击打。在后一种情形中,所有球都应被摆回并且活球应和违规前它是的相同,也就是说:\\
    If a striker fouls any ball including the cue-ball prior to striking it, the appropriate penalty will be imposed. The non-offender may then elect to play themselves from the position left, or request the offender to play again from the position left or the original position. In the latter case, all balls shall be replaced and the ball on shall be the same as it was prior to the infringement, namely:
    \begin{enumerate}[label=(\roman*)]
        \item 任何红球,当红球之前是活球时;\\
        any Red, where Red was the ball on;
        \item 成为活[球]的彩球,当所有红球之前都离开台面时;\\
        the colour on, where all the Reds were off the table;
        \item 一颗依击球手选择的彩球,当活球之前是已被击球入袋的一颗红球或一颗被指定当成红球的自由球之后的一颗彩球时。\\
        a colour of the striker's choice, where the ball on was a colour after a Red, or a free ball nominated as a Red had been potted.
    \end{enumerate}
    当将球摆回的决定被做出时磋商时间开始。\\
    A consultation period starts when the decision is made to replace the ball(s).
    \item 如果当母球处于手中球状态时一颗处于比赛中状态的目标球被击球手碰到,那么裁判应宣告``犯规''并且对下一次击球而言母球会保持手中球状态,除非当母球不在击球手的控制中时犯规被造成并与其有关。\\
    If an object ball in play is disturbed by the striker while the cue-ball is in-hand, the referee shall call FOUL and the cue-ball will remain in-hand for the next stroke, unless the foul is committed involving the cue-ball while it is not in the striker's possession.
\end{enumerate}

\subsection{罚分}\label{22311}

\noindent 下列行为是犯规并招致四罚分除非在下面的段落\ref{22311a}至\ref{22311d}中更高的罚分被指明。\\
The following acts are fouls and incur four penalty points unless higher penalty points are indicated in paragraphs \ref{22311a} to \ref{22311d} below.
\begin{enumerate}[label=(\alph*),start=1]
    \item \label{22311a}活球的分值因:\\
    Value of the ball on by:
    \begin{enumerate}[label=(\roman*)]
        \item \label{22311ai}在裁判已完成将一颗被当成自由球的彩球摆上点位前击球;\\
        striking before the referee has completed the spotting of a colour taken as a free ball;
        \item 在一次击球中击打母球多于一次;\\
        striking the cue-ball more than once during a stroke;
        \item \label{22311aiii}当双脚都离开地面时击球;\\
        striking when both feet are off the floor;
        \item 在斯诺克双打中在[正确的]击球轮外击打;\\
        playing out of turn in Four-handed Snooker;
        \item 不恰当地从手中球状态开始击打,包括开局击打时;\\
        playing improperly from in-hand, including at the opening stroke;
        \item 造成母球未接触任何目标球;\\
        causing the cue-ball to fail to contact any object ball;
        \item 造成母球掉袋;\\
        causing the cue-ball to be pocketed;
        \item 造成母球被被指定的自由球做斯诺克,除如第\ref{223}节规则\ref{22312}\ref{22312b}\ref{22312bii}中所述的外;\\
        causing the cue-ball to be snookered by the nominated free ball, except as provided for in Section \ref{223} Rule \ref{22312}\ref{22312b}\ref{22312bii};
        \item 在六红球斯诺克中造成母球被被指定的球做斯诺克;\\
        causing the cue-ball to be snookered by the nominated ball in Six Reds Snooker;
        \item 打一次跳球;\\
        playing a jump shot;
        \item 用不标准的球杆比赛;或\\
        playing with a non-standard cue; or
        \item 违反第\ref{223}节规则\ref{22318}\ref{22318e}同搭档商讨或交流。\\
        conferring or communicating with a partner contrary to Section \ref{223} Rule \ref{22318}\ref{22318e}.
    \end{enumerate}
    \item \label{22311b}活球或所涉及的球的分值中更高的因:\\
    Value of the ball on or ball concerned, whichever is higher, by:
    \begin{enumerate}[label=(\roman*)]
        \item 当任意球未静止时击球;\\
        striking when any ball is not at rest;
        \item \label{22311bii}在裁判已完成将一颗不是自由球的彩球摆上点位前击球;\\
        striking before the referee has completed the spotting of a colour that is not a free ball;
        \item 造成非活球掉袋;\\
        causing a ball not on to be pocketed;
        \item 造成母球首先击中非活球或当一颗自由球被指定时造成母球首先击中任意不是被指定的自由球的球除非它和一颗活球被同时击中。\\
        causing the cue-ball to first hit a ball not on or, when a free ball is nominated, causing the cue-ball to first hit any ball other than the nominated free ball unless it was hit simultaneously with a ball on;
        \item 造成一次推击;\\
        making a push stroke;
        \item 以球员的身体、衣服或装备的任意部分接触一颗处于比赛中状态的球或任意用于标注一颗处于比赛中状态的球的装置;\\
        contacting, with any part of the player's person, attire or equipment, a ball in play, or any device used to mark a ball in play;
        \item 当母球处于手中球状态时,以母球接触一颗处于比赛中状态的球;\\
        contacting a ball in play with the cue-ball, while the cue-ball is in-hand;
        \item 造成一颗处于比赛状态中的球接触任意当前击球轮中或之前击球轮中留在球桌附近或台面上的物体或装备;\\
        causing a ball in play to contact any object or equipment left at or on the table during the turn or from previous turns;
        \item 在任意被移除去清洁的球已被摆回台面前击球;或\\
        striking before any balls removed for cleaning have been returned to the table; or
        \item 造成一颗球被迫离台面。\\
        causing a ball to be forced off the table.
    \end{enumerate}
    \item 活球的分值或两颗所涉及的球中更高的分值因造成母球首先同时击中不是两颗红球(当红球是活球时)或一颗被指定的自由球和一颗活球的两颗球。\\
    Value of the ball on or higher value of the two balls concerned by causing the cue-ball to first hit simultaneously two balls, other than two Reds (when Red is the ball on) or a nominated free ball and a ball on.
    \item \label{22311d}七分如果击球手:\\
    Seven points if the striker:
    \begin{enumerate}[label=(\roman*)]
        \item 为任意目的使用离开台面的球;\\
        uses a ball off the table for any purpose;
        \item 使用任意物体测量空档或距离;\\
        uses any object to measure gaps or distance;
        \item 在连续的击球中击打多颗红球或一颗被指定的自由球后的一颗红球;\\
        plays at Reds, or a nominated free ball followed by a Red, in successive strokes;
        \item 在某局开始后将除白球外的任意球用作母球;\\
        uses any ball other than White as the cue-ball after the frame has started;
        \item 未指定他们以为活[球]的球,在被裁判要求如此做时;或\\
        fails to declare which ball they are on when requested to do so by the referee; or
        \item 在将红球(或被指定当成红球的自由球)击球入袋后,在彩球已被指定前造成犯规。\\
        after potting a Red (or free ball nominated as a Red), commits a foul before a colour has been nominated.
    \end{enumerate}
\end{enumerate}
\noindent 下列行为是受罚并招致四罚分除非在下面的段落\ref{22311e}至\ref{22311g}中更高的罚分被指明。\\
The following acts are penalties and incur four penalty points unless higher penalty points are indicated in paragraphs \ref{22311e} to \ref{22311g} below.
\begin{enumerate}[label=(\alph*),start=5]
    \item \label{22311e}活球或所涉及的球的分值中更高的因如第\ref{223}节规则\ref{2233}\ref{2233k} 中所述在击球轮外造成违规。\\
    Value of the ball on or ball concerned, whichever is higher by committing an infringement, out of turn, as described in Section \ref{223} Rule \ref{2233}\ref{2233k}.
    \item 七分如果在磋商时间内任意球员以他们身体、衣服或装备的任意部分接触任意在比赛区域中的球。\\
    Seven points if any player contacts, with any part of their person, attire or equipment, any ball on the playing area during a consultation period.
    \item \label{22311g}七分如果非击球手:\\
    Seven points if the non-striker:
    \begin{enumerate}[label=(\roman*)]
        \item 为任意目的使用离开台面的球;或\\
        uses a ball off the table for any purpose; or
        \item 使用任意物体测量空档或距离;\\
        uses any object to measure gaps or distance.
    \end{enumerate}
\end{enumerate}

\subsection{犯规后被做斯诺克}\label{22312}

\noindent 在一次犯规后,如果母球被做斯诺克(参见第\ref{222}节规则\ref{22217}),那么裁判应宣告``自由球''。\\
After a foul, if the cue-ball is snookered (see Section \ref{222} Rule \ref{22217}), the referee shall call FREE BALL.
\begin{enumerate}[label=(\alph*)]
    \item 如果下一个在击球轮中的球员选择[自己]进行下一次击球,那么:\\
    If the player next in turn elects to play the next stroke:
    \begin{enumerate}[label=(\roman*)]
        \item 他们可以指定任意球当成活球,但自由球不能是活球;\\
        they may nominate any ball as the ball on, but a free ball cannot be the ball on;
        \item 任何被指定的自由球都应被视为并赋予分值成活球但区别在于若被击球入袋则其应随后被摆上点位。\\
        any nominated free ball shall be regarded as, and acquire the value of, the ball on except that, if potted, it shall then be spotted.
    \end{enumerate}
    \item \label{22312b}那么是犯规,如果母球:\\
    It is a foul if the cue-ball should:
    \begin{enumerate}[label=(\roman*)]
        \item 未能首先击中被指定的自由球除非它和一颗活球被同时击中;或\\
        fail to hit the nominated free ball first unless it was hit simultaneously with a ball on; or
        \item \label{22312bii}在一次没有得分的击打后被被指定的自由球对所有红球或活球做斯诺克,除当粉球和黑球是台面上仅剩的目标球时。\\
        after a non-scoring stroke, be snookered on all Reds or the ball on by the nominated free ball, except when the Pink and Black are the only object balls remaining on the table.
    \end{enumerate}
    \item 如果被指定的自由球被击球入袋,那么它被摆上点位并且活球的分值被记录。\\
    If the nominated free ball is potted, it is spotted and the value of the ball on is scored.
    \item 如果一颗活球在母球首先击中被指定的自由球或同时击中[此球]和一颗活球后被击球入袋,那么活球应被计分并保持离开台面。\\
    If a ball on is potted, after the cue-ball hit the nominated free ball first, or simultaneously with a ball on, the ball on is scored and remains off the table.
    \item 如果被指定的自由球和一颗活球都被击球入袋,那么只有活球被计分除非它是红球此时每颗被击球入袋的球都计分。被指定的自由球被随后摆上点位并且活球保持离开台面。\\
    If both the nominated free ball and a ball on are potted, only the ball on is scored unless it was a Red, when each ball potted is scored. The nominated free ball is then spotted and the ball on remains off the table.
    \item 如果违规者被要求再次击打,或球摆回的要求被非违规者做出(如第\ref{223}节规则\ref{22310}\ref{22310i}、\ref{22314}\ref{22314b}、\ref{22314}\ref{22314e}和\ref{22316}中[所述]),那么自由球的选项变得无效。\\
    If the offender is asked to play again, or a request is made by the non-offender for the replacement of the ball(s) (as in Section \ref{223} Rules \ref{22310}\ref{22310i}, \ref{22314}\ref{22314b}, \ref{22314}\ref{22314e} and \ref{22316}), the free ball option becomes void.
\end{enumerate}

\subsection{再次击打}

\noindent 一旦在一次犯规后一位球员已要求一位对手再次击打或在一次犯规或一次犯规且未尽力后已要求球摆回,这个要求就不可撤回。已被要求再次击打的违规者有权:\\
Once a player has requested an opponent to play again after a foul or requested the replacement of ball(s) after a foul or a Foul and a Miss, such request cannot be withdrawn. The offender, having been asked to play again, is entitled to:
\begin{enumerate}[label=(\alph*)]
    \item 改变他们的主意,对:\\
    change their mind as to:
    \begin{enumerate}[label=(\roman*)]
        \item 哪种击球他们会进行;和\\
        which stroke they will play; and
        \item 哪颗活球他们会尝试击中。\\
        which ball on they will attempt to hit.
    \end{enumerate}
    \item 因任意他们可以击球入袋的球得分。\\
    score points for any ball(s) they may pot.
\end{enumerate}

\subsection{犯规且未尽力}\label{22314}

\begin{enumerate}[label=(\alph*)]
    \item \label{22314a}击球手应以他们的最大能力尽力击中活球或在一颗红球或一颗被指定当成红球的自由球已被击球入袋后的可以是活[球]的球。如果裁判认为此规则被违反\footnote{在职业比赛中通常只要击球手未能击中活球或在一颗红球或一颗被指定当成红球的自由球已被击球入袋后的可以是活球的球裁判就认为此规则被违反,除非击中活球或在一颗红球或一颗被指定当成红球的自由球已被击球入袋后的可以是活球的球确实十分困难。在其他比赛中裁判的判罚可以宽松。},那么他们应宣告``犯规且未尽力''\footnote{``犯规且未尽力''也称``无意识救球''。}除非:\\
    The striker shall, to the best of their ability, endeavour to hit the ball on or a ball that could be on after a Red, or a free ball nominated as a Red, has been potted. If the referee considers the Rule infringed, they shall call FOUL AND A MISS unless:
    \begin{enumerate}[label=(\roman*)]
        \item \label{22314ai}任意球员在被进行的击球前需要罚分[才能获胜]或由于被进行的击球而需要罚分[才能获胜]并且裁判确信未尽力不是故意的;\\
        any player required penalty points before, or as a result of, the stroke being played and the referee is satisfied that the miss was not intentional;
        \item 不可能击中活球的情况成为实际。在后一种情形中前提必须是击球手正尝试击中活球,只要他们向活球用足够的力量直接地或间接地击打且在裁判看来如果没有阻挡的球便已触及活球。\\
        a situation exists where it is impossible to hit the ball on. In the latter case it must be assumed the striker is attempting to hit the ball on provided that they play, directly or indirectly, at the ball on with sufficient strength, in the referee's opinion, to have reached the ball on but for the obstructing ball(s).
    \end{enumerate}
    \item \label{22314b}在一次``犯规且未尽力''已被宣告后,非违规者可以要求违规者从留存的状态或原来的状态再次击打,且在后一种情形中所有的球都应被摆回并且活球应和最后一次被进行的击球前它是的相同,也就是说:\\
    After a FOUL AND A MISS has been called, the non-offender may request the offender to play again from the position left or the original position, in which latter case all balls shall be replaced and the ball on shall be the same as it was prior to the last stroke made, namely:
    \begin{enumerate}[label=(\roman*)]
        \item 任何红球,当红球之前是活球时;\\
        any Red, where Red was the ball on;
        \item 成为活[球]的彩球,当所有红球之前都离开台面时;或\\
        the colour on, where all the Reds were off the table; or
        \item 一颗依击球手选择的彩球,当活球之前是已被击球入袋的一颗红球或一颗被指定当成红球的自由球之后的一颗彩球时。\\
        a colour of the striker's choice, where the ball on was a colour after a Red, or a free ball nominated as a Red had been potted.
    \end{enumerate}
    \item \label{22314c}当有一条从母球到任意是或可以是活[球]的球的清晰的直线线路时如果击球手在进行一次击球时未能首先击中一颗活球,那么裁判应宣告``犯规且未尽力''除非如第\ref{223}节规则\ref{22314}\ref{22314a}\ref{22314ai}中所述。\\
    If the striker, in making a stroke, fails to first hit a ball on when there is a clear path in a straight line from the cue-ball to any part of any ball that is or could be on, the referee shall call FOUL AND A MISS unless as described under Section \ref{223} Rule \ref{22314}\ref{22314a}\ref{22314ai}.
    \item \label{22314d}当有一条从母球到任意是或可以是活[球]的球的清晰的直线线路,以致中心的,整球的,接触是可行的(在红球[是活球]的情形中,这被理解成任意红球的不被彩球阻挡的整个直径[是可看到的]),或母球与可以是活球的球相贴时,在一次``犯规且未尽力''\footnote{此时的``犯规且未尽力''也称``无意识击球''。}已根据上面段落\ref{22314c}被宣告后:\\
    After a FOUL AND A MISS has been called under paragraph \ref{22314c} above when there was a clear path in a straight line from the cue-ball to a ball that was on or could have been on, such that central, full ball, contact was available (in the case of Reds, this to be taken as a full diameter of any Red that is not obstructed by a colour), or when the cue-ball is touching a ball that could be on, then:
    \begin{enumerate}[label=(\roman*)]
        \item \label{22314di}从原来的状态进行的击球中的第二次首先击中活球的失败应被无视分差地宣告``犯规且未尽力'';\\
        a second failure to first hit a ball on in making a stroke from the original position shall be called as a FOUL AND A MISS regardless of the difference in scores;
        \item \label{22314dii}如果是如上面\ref{22314di}中[所述]的第二次失败,那么若被要求从原来的状态再次击打,则违规者应被裁判警告接下来的失败会导致此局被判定他们的对方赢得。然而,如果警告未被宣布那么一局不能被判定赢得。如果裁判未宣布警告,那么只要``犯规且未尽力''的宣告序列已继续,击球手就会被在第一个可行的机会警告;\\
        in the event of a second failure as in \ref{22314di} above, if asked to play again from the original position, the offender shall be Warned by the referee that a further failure will result in the frame being awarded to their opponent. However, a frame cannot be awarded if a Warning has not been issued. If the referee has not issued the Warning, provided the sequence of FOUL AND A MISS calls has continued, the striker will be Warned at the first available opportunity;
        \item 如果被要求从留存的状态击打,那么如\ref{22314d}\ref{22314di}和\ref{22314d} \ref{22314dii}中[所述]的``犯规且未尽力''序列终止。\\
        if asked to play from the position left, the Foul and a Miss sequence as in \ref{22314d}\ref{22314di} and \ref{22314d}\ref{22314dii} ends.
    \end{enumerate}
    \item \label{22314e}在所有球都已被根据本规则摆回且击球手对包括母球在内的任意球犯规后,如果击球还未被进行那么未尽力不会被宣告。在此情形中适当的处罚会被推行。非违规者可以随后选择自己从留存的状态击打,或要求违规者从留存的状态或原来的状态再次击打,且在后一种情形中所有球都应被摆回并且活球应和最后一次被进行的击球前它是的相同,也就是说:\\
    After all balls have been replaced under this Rule, and the striker fouls any ball, including the cue-ball, a MISS will not be called if a stroke has not been played. In this case the appropriate penalty will be imposed. The non-offender may then elect to play themselves from the position left, or request the offender to play again from the position left or the original position, in which latter case all balls shall be replaced and the ball on shall be the same as it was prior to the last stroke made, namely:
    \begin{enumerate}[label=(\roman*)]
        \item 任何红球,当红球之前是活球时;\\
        any Red, where Red was the ball on;
        \item 成为活[球]的彩球,当所有红球之前都离开台面时;或\\
        the colour on, where all the Reds were off the table; or
        \item 一颗依击球手选择的彩球,当活球之前是已被击球入袋的一颗红球或一颗被指定当成红球的自由球之后的一颗彩球时。\\
        a colour of the striker's choice, where the ball on was a colour after a Red, or a free ball nominated as a Red had been potted.
    \end{enumerate}
    如果上面的情况发生在如上面段落\ref{22314d}中所述的``犯规且未尽力''的宣告序列中,那么任何涉及可能的判定此局他们的对方赢得的警告都应只在所有球都已被摆回违规前它们原来的位置时才仍有效。\\
    If the above situation arises during a sequence of FOUL AND A MISS calls as described under paragraph \ref{22314d} above, any Warning concerning the possible awarding of the frame to their opponent shall only remain in effect when all balls have been replaced to their original position prior to the infringement.
    \item 如果在一次``犯规且未尽力''被宣告后从原来的状态击打的要求被做出,那么磋商时间开始。\\
    If, after a FOUL AND A MISS has been called, the request is made to play from the original position, a consultation period starts.
\end{enumerate}

\subsection{不因击球手而被移动的球}\label{22315}

\noindent 如果一颗静止的或正运动的球被碰到但不因击球手,那么它应被裁判摆回他们认为球本在的或本会静止到的位置且不处罚击球手。\\
If a ball, stationary or moving, is disturbed other than by the striker, it shall be replaced by the referee to the position they deem the ball was, or would have come to rest, without penalising the striker.

\noindent 当将球摆回的决定被做出时磋商时间开始。\\
A consultation period starts when the decision is made to replace the ball(s).
\begin{enumerate}[label=(\alph*)]
    \item 本规则应包括不是击球者的搭档的另一干扰或人员造成击球手移动球的情况,但不会适用于球因球桌表面的任何负面效应而移动的情况,除在下一次击球已被进行前被摆上点位的球移动的情况外。\\
    This Rule shall include cases where another occurrence or person, other than the striker's partner causes the striker to move a ball, but will not apply in cases where a ball moves due to any defect in the table surface, except in the case where a spotted ball moves before the next stroke has been made.
    \item 所有球员都不应因任何被裁判造成的对球的碰触被处罚。\\
    No player shall be penalised for any disturbance of balls by the referee.
\end{enumerate}

\subsection{被故意移动的球}\label{22316}

\noindent 除击打母球以进行一次击球,或在磋商时间内接触球外,如果任意球被击球手故意从比赛区域中移动或拿起\footnote{此时裁判可根据第\ref{224}节中的规则做出相应判罚。},那么裁判应宣告``犯规''。\\
Other than striking the cue-ball to make a stroke, or contacting a ball during a consultation period, if any ball is intentionally moved or picked up by the striker from the playing area, the referee shall call FOUL.
\begin{enumerate}[label=(\alph*)]
    \item \label{22316a}在静止的球被故意移动或拿起的情况中,非违规者可以随后;\\
    In a situation where a stationary ball is intentionally 
    moved or picked up, the non-offender may then;
    \begin{enumerate}[label=(\roman*)]
        \item 选择自己或要求他们的对手从留存的状态击打。在此情况中,任何没有静止于比赛区域中的球都会被视为已被迫离台面。\\
        elect to play themselves or request their opponent to play from the position left. For this situation, any ball that did not come to rest on the playing area will be considered forced off the table.
        \item 选择让所有球都被摆回它们原来的位置然后自己击打或要求他们的对手再次击打。在后一种情形中活球应和违规前它是的相同,也就是说;\\
        elect to have all balls replaced to their original position and play themselves or request their opponent to play again. In the latter case the ball on shall be the same as it was prior to the infringement, namely;
        \begin{enumerate}[label=(\roman*)]
            \item 任何红球,当红球之前是活球时;\\
            any Red, where Red was the ball on;
            \item 成为活[球]的彩球,当所有红球之前都离开台面时;\\
            the colour on, where all the Reds were off the table;
            \item 一颗依击球手选择的彩球,当活球之前是已被击球入袋的一颗红球或一颗被指定当成红球的自由球之后的一颗彩球时。\\
            a colour of the striker's choice, where the ball on was a colour after a Red, or a free ball nominated as a Red had been potted.
        \end{enumerate}
        如果上面的情况发生在如第\ref{223}节规则\ref{22314}\ref{22314d}中所述的``犯规且未尽力''的宣告序列中并且让违规者再次击打的要求被做出,那么任何涉及可能的判定此局他们的对方赢得的警告都应仍有效。\\
        If the above situation arises during a sequence of FOUL AND A MISS calls as described under Section \ref{223} Rule \ref{22314}\ref{22314d} and the request is made for the offender to play again, any Warning concerning the possible awarding of the frame to their opponent shall remain in effect.
    \end{enumerate}
    \item \label{22316b}在正运动的球被故意移动或拿起的情况中,裁判应根据公平竞赛原则做出尽可能的最佳决定。\\
    In a situation where a moving ball is intentionally moved or picked up, the referee shall make the best possible decision in the interest of fair play.
    \item 如果击球手以违反体育精神的行为击打母球以进行一次击球,那么他们如第\ref{224}节规则\ref{2241}中所述可被警告并且非犯规方会有如上面\ref{22316a}和\ref{22316b}中所述的选择权。\\
    If the striker strikes the cue-ball to make a stroke as an act of Unsporting Conduct, they may be Warned as described in Section \ref{224}, Rule \ref{2241} and the non-offender will have the options as described under \ref{22316a} and \ref{22316a} above.
\end{enumerate}
\noindent 当将球摆回的决定被做出时磋商时间开始。\\
A consultation period starts when the decision is made to replace the ball(s).

\subsection{僵局}\label{22317}

\noindent 如果裁判认为僵局的状态成为实际或正被接近,或被双方球员指明,那么裁判应向球员们提供重开此局的立即选项。此过程习惯上被称为重摆。\\
If the referee thinks a position of stalemate exists, or is being approached, or is indicated by both players, the referee shall offer the players the immediate option of re-starting the frame. This process is commonly referred to as a re-rack.
\begin{enumerate}[label=(\alph*)]
    \item 如果任何球员反对,那么裁判应允许比赛继续但限制此情况必须在一定的时段内改变,通常是在每方再来三次击球后但仍按裁判的自行决定权。\\
    If any player objects, the referee shall allow play to continue with the proviso that the situation must change within a stated period, usually after three more strokes to each side but at the referee's discretion.
    \item 如果在一定的时段已过后情况仍基本不变,那么裁判应清零所有得分并且为一局的开启重摆所有球。\\
    If the situation remains basically unchanged after the stated period has expired, the referee shall nullify all scores and re-set all balls as for the start of a frame.
    \item 相同的球员应再次进行开局击打,基于第\ref{223}节规则\ref{2233}\ref{2233d} \ref{2233diii},且相同的被建立的比赛次序保持不变。\\
    The same player shall again make the opening stroke, subject to Section \ref{223} Rule \ref{2233}\ref{2233d}\ref{2233diii}, with the same established order of play being maintained.
    \item 如果僵局出现在如第\ref{223}节规则\ref{2234}\ref{2234b}中所述的重置黑球期间,那么只有黑球被摆回点位并且相同的球员再次进行开局击打。\\
    If a stalemate occurs during a re-spotted Black as described in Section \ref{223} Rule \ref{2234}\ref{2234b}, only the Black will be spotted with the same player to make the opening stroke.
\end{enumerate}

\subsection{斯诺克双打}\label{22318}

\begin{enumerate}[label=(\alph*)]
    \item 在双打比赛(四位球员分为两方且每方两位球员)中,每方应轮流开局并且比赛次序应被在每局的开始确定且当被如此确定时必须在整局内保持不变。\\
    In a four-handed game (four players constituting two sides of two players) each side shall start alternate frames and the order of play shall be determined at the start of each frame and, when so determined, must be maintained throughout that frame.
    \item 球员们可以在每一新局的开始改变比赛次序。\\
    Players may change the order of play at the start of each new frame.
    \item 如果一次犯规被造成并且再次击打的要求被做出,那么造成犯规的球员进行下一次击球并且比赛次序不变。如果``犯规''因在[正确的]击球轮外击打被宣告,那么违规者的搭档会失去一次击球轮,无论违规者是否被要求再次击打。\\
    If a foul is committed and a request to play again is made, the player who committed the foul plays the next stroke and the order of play is unchanged. If the FOUL was called for playing out of turn, the offender's partner will lose a turn, whether or not the offender is asked to play again.
    \item 当一局以平局结束时,第\ref{223}节规则\ref{2234}适用。如果重置黑球是必要的,那么进行第一次击球的一方有哪位球员将进行此次击球的选择权。比赛次序必须随后按此局中[此局的开始所确定的]延续。\\
    When a frame ends in a tie, Section \ref{223} Rule \ref{2234} applies. If a re-spotted Black is necessary, the side who play the first stroke have the choice of which player will make that stroke. The order of play must then continue as in the frame.
    \item \label{22318e}搭档间可以在一局中商讨或交流但是当[其中]一位是击球手且已经前往球台则不可以直到他们的击球轮已结束。\\
    Partners may confer or communicate during a frame but not while one is the striker and has approached the table until their turn has ended. 
    \item 如果击球手的搭档造成违规,那么击球手[也]会被视为违规者。\\
    If the striker's partner commits an infringement, the striker will be considered as the offender.
\end{enumerate}

\subsection{六红球斯诺克}

\noindent 在六红球斯诺克比赛中斯诺克比赛的官方规则适用但有下列不同:\\
In a Six Reds Snooker game the official Rules of the Game of 
Snooker apply with the following variations.
\begin{enumerate}[label=(\alph*)]
    \item 如果违规者被要求从原来的状态再次击打,那么不会有多于五次的连续的``犯规且未尽力''的宣告。\\
    There will be no more than five consecutive FOUL AND A MISS calls if the offender is requested to play again from the original position.
    \item 在第四次连续的``犯规且未尽力''的宣告后,裁判应警告违规球员如果``犯规且未尽力''再被宣告那么非违规者可以:\\
    After the fourth consecutive FOUL AND A MISS call, the referee shall Warn the offending player that should a FOUL AND A MISS be called again the non-offender may:
    \begin{enumerate}[label=(\roman*)]
        \item 自球已静止处击打;或\\
        play from where the balls have come to rest; or
        \item 要求他们的对手自球已静止处击打;或\\
        ask their opponent to play from where the balls have come to rest; or
        \item 自比赛区域中的任意位置击打母球,除非任意球员在最后一次被进行的击球前需要罚分[才能获胜]或由于最后一次被进行的击球而需要罚分[才能获胜]。如果此选项被选择,那么第\ref{223}节规则\ref{22312}不应适用。\\
        play the cue-ball from any position on the playing area, unless any player needed penalty points before, or as a result of, the last stroke being played. If this option is chosen, Section \ref{223} Rule \ref{22312} shall not apply.
    \end{enumerate}
    \item 如果在一次``犯规且未尽力''的宣告后违规者被要求从留存的状态击打,那么之前的``犯规且未尽力''序列终止。\\
    If, after a FOUL AND A MISS call, the offender is requested to play from the position left, the previous Foul and a Miss sequence ends.
    \item 在将一颗红球或一颗被指定当成红球的自由球击球入袋后,击球手禁止如第\ref{222}节规则\ref{22217}中所述地让他们的对手被在被指定的彩球后做斯诺克。\\
    After potting a Red, or a free ball nominated as a Red, the striker must not leave their opponent snookered behind the nominated colour as described in Section \ref{222} Rule \ref{22217}.
\end{enumerate}

\subsection{辅助器材的使用}\label{22320}

\noindent 放置和移除球桌附近的击球手可能用到的任何器材都是他们的责任。\\
It is the responsibility of the striker to both place and remove any 
equipment they may use at the table.
\begin{enumerate}[label=(\alph*)]
    \item 击球手对包括但不限于他们带去球桌的架杆和套筒的所有物品负责,无论是被他们拥有的还是被借来的(除非来自裁判),并且在使用这样的器材时他们会因任何被[他们]造成的犯规被处罚。\\
    The striker is responsible for all items including, but not limited to, rests and extensions that they bring to the table, whether owned by them or borrowed (except from the referee), and they will be penalised for any fouls made when using this equipment.
    \item \label{22320b}已被包括裁判在内的第三方提供的通常被在球桌附近找到的器材不是由击球手担责的。如果这样的器材被证明有问题并因此造成击球手接触一颗或多颗球那么不是犯规。裁判若有必要则会根据第\ref{223}节规则\ref{22315}将任何球摆回原位并且击球手若在一次单杆中则会被允许继续且无处罚。\\
    Equipment normally found at the table which has been provided by another party, including the referee, is not the responsibility of the striker. It is not a foul if this equipment should prove to be faulty and thereby cause the striker to contact a ball or balls. The referee will, if necessary, reposition any balls in accordance with Section \ref{223} Rule \ref{22315} and the striker, if in a break, will be allowed to continue without penalty.
\end{enumerate}

\subsection{规则解释}

\begin{enumerate}
    \item 诸条件可能需要本规则被适用于残障人员的方式中的调整。尤其是例如:\\
    Circumstances may necessitate adjustment in how these Rules are applied for persons with disabilities. In particular and for example:
    \begin{enumerate}
        \item 第\ref{223}节规则\ref{22311}\ref{22311a}\ref{22311aiii}不能被适用于坐轮椅的球员;\\并且\\
        Section \ref{223} Rule \ref{22311}\ref{22311a}\ref{22311aiii} cannot be applied to players in wheelchairs; and
        \item 如果球员无法分辨不同颜色,比如红色和绿色,那么他们在要求裁判后应被告知球的颜色或它的位置。\\
        a player, upon request to the referee, shall be told the colour of a ball or its position if they are unable to differentiate between colours as, for example, Red and Green.
    \end{enumerate}
    \item 当没有裁判时,对立的某球员或某方会被视为依本规则要求的裁判。\\
    When there is no referee, the opposing player or side will be regarded as such for the purpose of these Rules.
    \item 依本比赛的规则,斯诺克的简化版本可被用任意数量的红球进行。\\
    Under these Rules of the Game, a simplified form of snooker can be played with any number of Red balls.
\end{enumerate}
